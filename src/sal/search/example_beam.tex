-> i = 0
beams
Beam(prompt='Convert the point $(0,3)$ in rectangular coordinates to polar coordinates.  Enter your answer in the form $(r,\theta),$ where $r > 0$ and $0 \\le \theta < 2 \\pi.$', index=0, current_text='', next_texts=None, lookahead_texts=None, stop_reasons=None, best_scores=[], all_scores=[], previous_text=None, pruned=False, history=[], completed=False, completion_tokens=0)
Beam(prompt='<|start_header_id|>system<|end_header_id|>\n\nCutting Knowledge Date: December 2023\nToday Date: 26 Feb 2025\n\nSolve the following math problem efficiently and clearly:\n\n- For simple problems (2 steps or fewer):\nProvide a concise solution with minimal explanation.\n\n- For complex problems (3 steps or more):\nUse this step-by-step format:\n\n## Step 1: [Concise description]\n[Brief explanation and calculations]\n\n## Step 2: [Concise description]\n[Brief explanation and calculations]\n\n...\n\nRegardless of the approach, always conclude with:\n\nTherefore, the final answer is: $\\boxed{answer}$. I hope it is correct.\n\nWhere [answer] is just the final number or expression that solves the problem.<|eot_id|><|start_header_id|>user<|end_header_id|>\n\nConvert the point $(0,3)$ in rectangular coordinates to polar coordinates.  Enter your answer in the form $(r,\theta),$ where $r > 0$ and $0 \\le \theta < 2 \\pi.$<|eot_id|><|start_header_id|>assistant<|end_header_id|>\n\n', index=0, current_text='', next_texts=['## Step 1: Understand the conversion formulas\nTo convert from rectangular (Cartesian) coordinates $(x, y)$ to polar coordinates $(r, \\theta)$, we use the following formulas:\n$r = \\sqrt{x^2 + y^2}$ for the radial coordinate,\n$\\theta = \\arctan\\left(\\frac{y}{x}\\right)$ for the angular coordinate.\n\n'], lookahead_texts=['## Step 1: Understand the conversion formulas\nTo convert from rectangular (Cartesian) coordinates $(x, y)$ to polar coordinates $(r, \\theta)$, we use the following formulas:\n$r = \\sqrt{x^2 + y^2}$ for the radial coordinate,\n$\\theta = \\arctan\\left(\\frac{y}{x}\\right)$ for the angular coordinate.\n\n## Step 2: Apply the formulas to the given point\nGiven the point $(0, 3)$, we can substitute $x = 0$ and $y = 3$ into the formulas.\n\n## Step 3: Calculate the radial coordinate $r$\nUsing the formula $r = \\sqrt{x^2 + y^2}$, we get\n$r = \\sqrt{0^2 + 3^2} = \\sqrt{9} = 3$.\n\n'], stop_reasons=['\n\n'], best_scores=[0.0], all_scores=[], previous_text=None, pruned=False, history=[], completed=False, completion_tokens=0)

beam.next_texts
['## Step 1: Understand the conversion formulas\nTo convert from rectangular (Cartesian) coordinates $(x, y)$ to polar coordinates $(r, \\theta)$, we use the following formulas:\n$r = \\sqrt{x^2 + y^2}$ for the radial coordinate,\n$\\theta = \\arctan\\left(\\frac{y}{x}\\right)$ for the angular coordinate.\n\n']
beam.stop_reasons
['\n\n']
beam.lookahead_texts
['## Step 1: Understand the conversion formulas\nTo convert from rectangular (Cartesian) coordinates $(x, y)$ to polar coordinates $(r, \\theta)$, we use the following formulas:\n$r = \\sqrt{x^2 + y^2}$ for the radial coordinate,\n$\\theta = \\arctan\\left(\\frac{y}{x}\\right)$ for the angular coordinate.\n\n## Step 2: Apply the formulas to the given point\nGiven the point $(0, 3)$, we can substitute $x = 0$ and $y = 3$ into the formulas.\n\n## Step 3: Calculate the radial coordinate $r$\nUsing the formula $r = \\sqrt{x^2 + y^2}$, we get\n$r = \\sqrt{0^2 + 3^2} = \\sqrt{9} = 3$.\n\n']
beam.current_text
## Step 1: Understand the conversion formulas
To convert from rectangular (Cartesian) coordinates $(x, y)$ to polar coordinates $(r, \theta)$, we use the following formulas:
$r = \sqrt{x^2 + y^2}$ for the radial coordinate,
$\theta = \arctan\left(\frac{y}{x}\right)$ for the angular coordinate.


## Step 1: Understand the conversion formulas
To convert from rectangular (Cartesian) coordinates $(x, y)$ to polar coordinates $(r, \theta)$, we use the following formulas:
$r = \sqrt{x^2 + y^2}$ for the radial coordinate,
$\theta = \arctan\left(\frac{y}{x}\right)$ for the angular coordinate.


beam.completion_tokens
0
0
Beam(prompt='Convert the point $(0,3)$ in rectangular coordinates to polar coordinates.  Enter your answer in the form $(r,\theta),$ where $r > 0$ and $0 \\le \theta < 2 \\pi.$', index=1, current_text='', next_texts=None, lookahead_texts=None, stop_reasons=None, best_scores=[], all_scores=[], previous_text=None, pruned=False, history=[], completed=False, completion_tokens=0)
Beam(prompt='<|start_header_id|>system<|end_header_id|>\n\nCutting Knowledge Date: December 2023\nToday Date: 26 Feb 2025\n\nSolve the following math problem efficiently and clearly:\n\n- For simple problems (2 steps or fewer):\nProvide a concise solution with minimal explanation.\n\n- For complex problems (3 steps or more):\nUse this step-by-step format:\n\n## Step 1: [Concise description]\n[Brief explanation and calculations]\n\n## Step 2: [Concise description]\n[Brief explanation and calculations]\n\n...\n\nRegardless of the approach, always conclude with:\n\nTherefore, the final answer is: $\\boxed{answer}$. I hope it is correct.\n\nWhere [answer] is just the final number or expression that solves the problem.<|eot_id|><|start_header_id|>user<|end_header_id|>\n\nConvert the point $(0,3)$ in rectangular coordinates to polar coordinates.  Enter your answer in the form $(r,\theta),$ where $r > 0$ and $0 \\le \theta < 2 \\pi.$<|eot_id|><|start_header_id|>assistant<|end_header_id|>\n\n', index=1, current_text='', next_texts=['## Step 1: Recall the formulas for converting rectangular coordinates to polar coordinates\nThe relationship between rectangular coordinates $(x,y)$ and polar coordinates $(r,heta)$ is given by $x = r\\cos(heta)$ and $y = r\\sin(heta)$.\n\n'], lookahead_texts=['## Step 1: Recall the formulas for converting rectangular coordinates to polar coordinates\nThe relationship between rectangular coordinates $(x,y)$ and polar coordinates $(r,heta)$ is given by $x = r\\cos(heta)$ and $y = r\\sin(heta)$.\n\n## Step 2: Apply the formulas to the given point (0,3)\nWe substitute $x = 0$ and $y = 3$ into the formulas to get $0 = r\\cos(heta)$ and $3 = r\\sin(heta)$.\n\n## Step 3: Solve for r\nWe can solve for $r$ by dividing both sides of the second equation by $\\sin(heta)$, giving $r = \\frac{3}{\\sin(heta)}$.\n\n'], stop_reasons=['\n\n'], best_scores=[0.0], all_scores=[], previous_text=None, pruned=False, history=[], completed=False, completion_tokens=0)

beam.next_texts
['## Step 1: Recall the formulas for converting rectangular coordinates to polar coordinates\nThe relationship between rectangular coordinates $(x,y)$ and polar coordinates $(r,heta)$ is given by $x = r\\cos(heta)$ and $y = r\\sin(heta)$.\n\n']
beam.stop_reasons
['\n\n']
beam.lookahead_texts
['## Step 1: Recall the formulas for converting rectangular coordinates to polar coordinates\nThe relationship between rectangular coordinates $(x,y)$ and polar coordinates $(r,heta)$ is given by $x = r\\cos(heta)$ and $y = r\\sin(heta)$.\n\n## Step 2: Apply the formulas to the given point (0,3)\nWe substitute $x = 0$ and $y = 3$ into the formulas to get $0 = r\\cos(heta)$ and $3 = r\\sin(heta)$.\n\n## Step 3: Solve for r\nWe can solve for $r$ by dividing both sides of the second equation by $\\sin(heta)$, giving $r = \\frac{3}{\\sin(heta)}$.\n\n']
beam.current_text
## Step 1: Recall the formulas for converting rectangular coordinates to polar coordinates
The relationship between rectangular coordinates $(x,y)$ and polar coordinates $(r,heta)$ is given by $x = r\cos(heta)$ and $y = r\sin(heta)$.


## Step 1: Recall the formulas for converting rectangular coordinates to polar coordinates
The relationship between rectangular coordinates $(x,y)$ and polar coordinates $(r,heta)$ is given by $x = r\cos(heta)$ and $y = r\sin(heta)$.


beam.completion_tokens
0
0
completions
[['## Step 1: Understand the conversion formulas\nTo convert from rectangular (Cartesian) coordinates $(x, y)$ to polar coordinates $(r, \\theta)$, we use the following formulas:\n$r = \\sqrt{x^2 + y^2}$ for the radial coordinate,\n$\\theta = \\arctan\\left(\\frac{y}{x}\\right)$ for the angular coordinate.\n\n'], ['## Step 1: Recall the formulas for converting rectangular coordinates to polar coordinates\nThe relationship between rectangular coordinates $(x,y)$ and polar coordinates $(r,heta)$ is given by $x = r\\cos(heta)$ and $y = r\\sin(heta)$.\n\n']]
scores
[[[0.99609375, 0.99609375]], [[0.99609375, 0.99609375]]]
[[0.99609375], [0.99609375]]

-> i = 1
beams
Beam(prompt='Convert the point $(0,3)$ in rectangular coordinates to polar coordinates.  Enter your answer in the form $(r,\theta),$ where $r > 0$ and $0 \\le \theta < 2 \\pi.$', index=0, current_text='## Step 1: Understand the conversion formulas\nTo convert from rectangular (Cartesian) coordinates $(x, y)$ to polar coordinates $(r, \\theta)$, we use the following formulas:\n$r = \\sqrt{x^2 + y^2}$ for the radial coordinate,\n$\\theta = \\arctan\\left(\\frac{y}{x}\\right)$ for the angular coordinate.\n\n', next_texts=['## Step 1: Understand the conversion formulas\nTo convert from rectangular (Cartesian) coordinates $(x, y)$ to polar coordinates $(r, \\theta)$, we use the following formulas:\n$r = \\sqrt{x^2 + y^2}$ for the radial coordinate,\n$\\theta = \\arctan\\left(\\frac{y}{x}\\right)$ for the angular coordinate.\n\n'], lookahead_texts=['## Step 1: Understand the conversion formulas\nTo convert from rectangular (Cartesian) coordinates $(x, y)$ to polar coordinates $(r, \\theta)$, we use the following formulas:\n$r = \\sqrt{x^2 + y^2}$ for the radial coordinate,\n$\\theta = \\arctan\\left(\\frac{y}{x}\\right)$ for the angular coordinate.\n\n## Step 2: Apply the formulas to the given point\nGiven the point $(0, 3)$, we can substitute $x = 0$ and $y = 3$ into the formulas.\n\n## Step 3: Calculate the radial coordinate $r$\nUsing the formula $r = \\sqrt{x^2 + y^2}$, we get\n$r = \\sqrt{0^2 + 3^2} = \\sqrt{9} = 3$.\n\n'], stop_reasons=['\n\n'], best_scores=[], all_scores=[0.99609375, 0.99609375], previous_text=None, pruned=False, history=['## Step 1: Understand the conversion formulas\nTo convert from rectangular (Cartesian) coordinates $(x, y)$ to polar coordinates $(r, \\theta)$, we use the following formulas:\n$r = \\sqrt{x^2 + y^2}$ for the radial coordinate,\n$\\theta = \\arctan\\left(\\frac{y}{x}\\right)$ for the angular coordinate.\n\n'], completed=False, completion_tokens=0)
Beam(prompt='<|start_header_id|>system<|end_header_id|>\n\nCutting Knowledge Date: December 2023\nToday Date: 26 Feb 2025\n\nSolve the following math problem efficiently and clearly:\n\n- For simple problems (2 steps or fewer):\nProvide a concise solution with minimal explanation.\n\n- For complex problems (3 steps or more):\nUse this step-by-step format:\n\n## Step 1: [Concise description]\n[Brief explanation and calculations]\n\n## Step 2: [Concise description]\n[Brief explanation and calculations]\n\n...\n\nRegardless of the approach, always conclude with:\n\nTherefore, the final answer is: $\\boxed{answer}$. I hope it is correct.\n\nWhere [answer] is just the final number or expression that solves the problem.<|eot_id|><|start_header_id|>user<|end_header_id|>\n\nConvert the point $(0,3)$ in rectangular coordinates to polar coordinates.  Enter your answer in the form $(r,\theta),$ where $r > 0$ and $0 \\le \theta < 2 \\pi.$<|eot_id|><|start_header_id|>assistant<|end_header_id|>\n\n## Step 1: Understand the conversion formulas\nTo convert from rectangular (Cartesian) coordinates $(x, y)$ to polar coordinates $(r, \\theta)$, we use the following formulas:\n$r = \\sqrt{x^2 + y^2}$ for the radial coordinate,\n$\\theta = \\arctan\\left(\\frac{y}{x}\\right)$ for the angular coordinate.\n\n', index=0, current_text='', next_texts=['## Step 2: Calculate the radial coordinate $r$\nUsing the formula $r = \\sqrt{x^2 + y^2}$, we substitute $x = 0$ and $y = 3$:\n$r = \\sqrt{0^2 + 3^2} = \\sqrt{0 + 9} = \\sqrt{9} = 3$.\n\n'], lookahead_texts=["## Step 2: Calculate the radial coordinate $r$\nUsing the formula $r = \\sqrt{x^2 + y^2}$, we substitute $x = 0$ and $y = 3$:\n$r = \\sqrt{0^2 + 3^2} = \\sqrt{0 + 9} = \\sqrt{9} = 3$.\n\n## Step 3: Calculate the angular coordinate $\\theta$\nUsing the formula $\\theta = \\arctan\\left(\\frac{y}{x}\\right)$, we substitute $x = 0$ and $y = 3$:\n$\\theta = \\arctan\\left(\\frac{3}{0}\\right)$. However, since $\\arctan(0)$ is undefined, we must use the fact that $\\arctan(0) = \\frac{\\pi}{2}$ and the fact that $\\arctan(x)$ is an odd function, so $\\arctan\\left(\\frac{3}{0}\\right) = \\frac{\\pi}{2} - \\arctan(3)$.\n\n## Step 4: Calculate $\\arctan(3)$\nTo calculate $\\arctan(3)$, we can use a calculator or the fact that $\\arctan(x) = \\frac{\\pi}{2} - \\arctan\\left(\\frac{1}{x}\\right)$. Let's use the latter approach:\n$\\arctan(3) = \\frac{\\pi}{2} - \\arctan\\left(\\frac{1}{3}\\right)$.\n\n"], stop_reasons=['\n\n'], best_scores=[0.0], all_scores=[], previous_text=None, pruned=False, history=[], completed=False, completion_tokens=0)

beam.next_texts
['## Step 2: Calculate the radial coordinate $r$\nUsing the formula $r = \\sqrt{x^2 + y^2}$, we substitute $x = 0$ and $y = 3$:\n$r = \\sqrt{0^2 + 3^2} = \\sqrt{0 + 9} = \\sqrt{9} = 3$.\n\n']
beam.stop_reasons
['\n\n']
beam.lookahead_texts
["## Step 2: Calculate the radial coordinate $r$\nUsing the formula $r = \\sqrt{x^2 + y^2}$, we substitute $x = 0$ and $y = 3$:\n$r = \\sqrt{0^2 + 3^2} = \\sqrt{0 + 9} = \\sqrt{9} = 3$.\n\n## Step 3: Calculate the angular coordinate $\\theta$\nUsing the formula $\\theta = \\arctan\\left(\\frac{y}{x}\\right)$, we substitute $x = 0$ and $y = 3$:\n$\\theta = \\arctan\\left(\\frac{3}{0}\\right)$. However, since $\\arctan(0)$ is undefined, we must use the fact that $\\arctan(0) = \\frac{\\pi}{2}$ and the fact that $\\arctan(x)$ is an odd function, so $\\arctan\\left(\\frac{3}{0}\\right) = \\frac{\\pi}{2} - \\arctan(3)$.\n\n## Step 4: Calculate $\\arctan(3)$\nTo calculate $\\arctan(3)$, we can use a calculator or the fact that $\\arctan(x) = \\frac{\\pi}{2} - \\arctan\\left(\\frac{1}{x}\\right)$. Let's use the latter approach:\n$\\arctan(3) = \\frac{\\pi}{2} - \\arctan\\left(\\frac{1}{3}\\right)$.\n\n"]
beam.current_text
## Step 2: Calculate the radial coordinate $r$
Using the formula $r = \sqrt{x^2 + y^2}$, we substitute $x = 0$ and $y = 3$:
$r = \sqrt{0^2 + 3^2} = \sqrt{0 + 9} = \sqrt{9} = 3$.


## Step 1: Understand the conversion formulas
To convert from rectangular (Cartesian) coordinates $(x, y)$ to polar coordinates $(r, \theta)$, we use the following formulas:
$r = \sqrt{x^2 + y^2}$ for the radial coordinate,
$\theta = \arctan\left(\frac{y}{x}\right)$ for the angular coordinate.

## Step 2: Calculate the radial coordinate $r$
Using the formula $r = \sqrt{x^2 + y^2}$, we substitute $x = 0$ and $y = 3$:
$r = \sqrt{0^2 + 3^2} = \sqrt{0 + 9} = \sqrt{9} = 3$.


beam.completion_tokens
0
0
Beam(prompt='Convert the point $(0,3)$ in rectangular coordinates to polar coordinates.  Enter your answer in the form $(r,\theta),$ where $r > 0$ and $0 \\le \theta < 2 \\pi.$', index=1, current_text='## Step 1: Recall the formulas for converting rectangular coordinates to polar coordinates\nThe relationship between rectangular coordinates $(x,y)$ and polar coordinates $(r,heta)$ is given by $x = r\\cos(heta)$ and $y = r\\sin(heta)$.\n\n', next_texts=['## Step 1: Recall the formulas for converting rectangular coordinates to polar coordinates\nThe relationship between rectangular coordinates $(x,y)$ and polar coordinates $(r,heta)$ is given by $x = r\\cos(heta)$ and $y = r\\sin(heta)$.\n\n'], lookahead_texts=['## Step 1: Recall the formulas for converting rectangular coordinates to polar coordinates\nThe relationship between rectangular coordinates $(x,y)$ and polar coordinates $(r,heta)$ is given by $x = r\\cos(heta)$ and $y = r\\sin(heta)$.\n\n## Step 2: Apply the formulas to the given point (0,3)\nWe substitute $x = 0$ and $y = 3$ into the formulas to get $0 = r\\cos(heta)$ and $3 = r\\sin(heta)$.\n\n## Step 3: Solve for r\nWe can solve for $r$ by dividing both sides of the second equation by $\\sin(heta)$, giving $r = \\frac{3}{\\sin(heta)}$.\n\n'], stop_reasons=['\n\n'], best_scores=[], all_scores=[0.99609375, 0.99609375], previous_text=None, pruned=False, history=['## Step 1: Recall the formulas for converting rectangular coordinates to polar coordinates\nThe relationship between rectangular coordinates $(x,y)$ and polar coordinates $(r,heta)$ is given by $x = r\\cos(heta)$ and $y = r\\sin(heta)$.\n\n'], completed=False, completion_tokens=0)
Beam(prompt='<|start_header_id|>system<|end_header_id|>\n\nCutting Knowledge Date: December 2023\nToday Date: 26 Feb 2025\n\nSolve the following math problem efficiently and clearly:\n\n- For simple problems (2 steps or fewer):\nProvide a concise solution with minimal explanation.\n\n- For complex problems (3 steps or more):\nUse this step-by-step format:\n\n## Step 1: [Concise description]\n[Brief explanation and calculations]\n\n## Step 2: [Concise description]\n[Brief explanation and calculations]\n\n...\n\nRegardless of the approach, always conclude with:\n\nTherefore, the final answer is: $\\boxed{answer}$. I hope it is correct.\n\nWhere [answer] is just the final number or expression that solves the problem.<|eot_id|><|start_header_id|>user<|end_header_id|>\n\nConvert the point $(0,3)$ in rectangular coordinates to polar coordinates.  Enter your answer in the form $(r,\theta),$ where $r > 0$ and $0 \\le \theta < 2 \\pi.$<|eot_id|><|start_header_id|>assistant<|end_header_id|>\n\n## Step 1: Recall the formulas for converting rectangular coordinates to polar coordinates\nThe relationship between rectangular coordinates $(x,y)$ and polar coordinates $(r,heta)$ is given by $x = r\\cos(heta)$ and $y = r\\sin(heta)$.\n\n', index=1, current_text='', next_texts=['## Step 2: Plug in the given rectangular coordinates into the formulas\nWe have $(x,y) = (0,3)$, so $x = 0 = r\\cos(heta)$ and $y = 3 = r\\sin(heta)$.\n\n'], lookahead_texts=['## Step 2: Plug in the given rectangular coordinates into the formulas\nWe have $(x,y) = (0,3)$, so $x = 0 = r\\cos(heta)$ and $y = 3 = r\\sin(heta)$.\n\n## Step 3: Solve for $r$ and $heta$\nFrom the equation $x = 0 = r\\cos(heta)$, we get $r = 0$ since $x$ cannot be zero. However, this would imply $heta = 0$, which is not possible since $0 \\le heta < 2\\pi$. This means we need to reconsider our approach.\n\n## Step 4: Reconsider the approach\nSince $x = 0$, we know that $r = 0$ is not a valid solution. We need to find another approach to solve for $r$ and $heta$.\n\n'], stop_reasons=['\n\n'], best_scores=[0.0], all_scores=[], previous_text=None, pruned=False, history=[], completed=False, completion_tokens=0)

beam.next_texts
['## Step 2: Plug in the given rectangular coordinates into the formulas\nWe have $(x,y) = (0,3)$, so $x = 0 = r\\cos(heta)$ and $y = 3 = r\\sin(heta)$.\n\n']
beam.stop_reasons
['\n\n']
beam.lookahead_texts
['## Step 2: Plug in the given rectangular coordinates into the formulas\nWe have $(x,y) = (0,3)$, so $x = 0 = r\\cos(heta)$ and $y = 3 = r\\sin(heta)$.\n\n## Step 3: Solve for $r$ and $heta$\nFrom the equation $x = 0 = r\\cos(heta)$, we get $r = 0$ since $x$ cannot be zero. However, this would imply $heta = 0$, which is not possible since $0 \\le heta < 2\\pi$. This means we need to reconsider our approach.\n\n## Step 4: Reconsider the approach\nSince $x = 0$, we know that $r = 0$ is not a valid solution. We need to find another approach to solve for $r$ and $heta$.\n\n']
beam.current_text
## Step 2: Plug in the given rectangular coordinates into the formulas
We have $(x,y) = (0,3)$, so $x = 0 = r\cos(heta)$ and $y = 3 = r\sin(heta)$.


## Step 1: Recall the formulas for converting rectangular coordinates to polar coordinates
The relationship between rectangular coordinates $(x,y)$ and polar coordinates $(r,heta)$ is given by $x = r\cos(heta)$ and $y = r\sin(heta)$.

## Step 2: Plug in the given rectangular coordinates into the formulas
We have $(x,y) = (0,3)$, so $x = 0 = r\cos(heta)$ and $y = 3 = r\sin(heta)$.


beam.completion_tokens
0
0
completions
[['## Step 1: Understand the conversion formulas\nTo convert from rectangular (Cartesian) coordinates $(x, y)$ to polar coordinates $(r, \\theta)$, we use the following formulas:\n$r = \\sqrt{x^2 + y^2}$ for the radial coordinate,\n$\\theta = \\arctan\\left(\\frac{y}{x}\\right)$ for the angular coordinate.\n\n## Step 2: Calculate the radial coordinate $r$\nUsing the formula $r = \\sqrt{x^2 + y^2}$, we substitute $x = 0$ and $y = 3$:\n$r = \\sqrt{0^2 + 3^2} = \\sqrt{0 + 9} = \\sqrt{9} = 3$.\n\n'], ['## Step 1: Recall the formulas for converting rectangular coordinates to polar coordinates\nThe relationship between rectangular coordinates $(x,y)$ and polar coordinates $(r,heta)$ is given by $x = r\\cos(heta)$ and $y = r\\sin(heta)$.\n\n## Step 2: Plug in the given rectangular coordinates into the formulas\nWe have $(x,y) = (0,3)$, so $x = 0 = r\\cos(heta)$ and $y = 3 = r\\sin(heta)$.\n\n']]
scores
[[[0.99609375, 1.0, 0.9609375]], [[0.99609375, 1.0, 0.96875]]]
[[0.9609375], [0.96875]]

-> i = 2
beams
Beam(prompt='Convert the point $(0,3)$ in rectangular coordinates to polar coordinates.  Enter your answer in the form $(r,\theta),$ where $r > 0$ and $0 \\le \theta < 2 \\pi.$', index=0, current_text='## Step 1: Understand the conversion formulas\nTo convert from rectangular (Cartesian) coordinates $(x, y)$ to polar coordinates $(r, \\theta)$, we use the following formulas:\n$r = \\sqrt{x^2 + y^2}$ for the radial coordinate,\n$\\theta = \\arctan\\left(\\frac{y}{x}\\right)$ for the angular coordinate.\n\n## Step 2: Calculate the radial coordinate $r$\nUsing the formula $r = \\sqrt{x^2 + y^2}$, we substitute $x = 0$ and $y = 3$:\n$r = \\sqrt{0^2 + 3^2} = \\sqrt{0 + 9} = \\sqrt{9} = 3$.\n\n', next_texts=['## Step 2: Calculate the radial coordinate $r$\nUsing the formula $r = \\sqrt{x^2 + y^2}$, we substitute $x = 0$ and $y = 3$:\n$r = \\sqrt{0^2 + 3^2} = \\sqrt{0 + 9} = \\sqrt{9} = 3$.\n\n'], lookahead_texts=["## Step 2: Calculate the radial coordinate $r$\nUsing the formula $r = \\sqrt{x^2 + y^2}$, we substitute $x = 0$ and $y = 3$:\n$r = \\sqrt{0^2 + 3^2} = \\sqrt{0 + 9} = \\sqrt{9} = 3$.\n\n## Step 3: Calculate the angular coordinate $\\theta$\nUsing the formula $\\theta = \\arctan\\left(\\frac{y}{x}\\right)$, we substitute $x = 0$ and $y = 3$:\n$\\theta = \\arctan\\left(\\frac{3}{0}\\right)$. However, since $\\arctan(0)$ is undefined, we must use the fact that $\\arctan(0) = \\frac{\\pi}{2}$ and the fact that $\\arctan(x)$ is an odd function, so $\\arctan\\left(\\frac{3}{0}\\right) = \\frac{\\pi}{2} - \\arctan(3)$.\n\n## Step 4: Calculate $\\arctan(3)$\nTo calculate $\\arctan(3)$, we can use a calculator or the fact that $\\arctan(x) = \\frac{\\pi}{2} - \\arctan\\left(\\frac{1}{x}\\right)$. Let's use the latter approach:\n$\\arctan(3) = \\frac{\\pi}{2} - \\arctan\\left(\\frac{1}{3}\\right)$.\n\n"], stop_reasons=['\n\n'], best_scores=[], all_scores=[0.99609375, 1.0, 0.9609375], previous_text=None, pruned=False, history=['## Step 1: Understand the conversion formulas\nTo convert from rectangular (Cartesian) coordinates $(x, y)$ to polar coordinates $(r, \\theta)$, we use the following formulas:\n$r = \\sqrt{x^2 + y^2}$ for the radial coordinate,\n$\\theta = \\arctan\\left(\\frac{y}{x}\\right)$ for the angular coordinate.\n\n', '## Step 2: Calculate the radial coordinate $r$\nUsing the formula $r = \\sqrt{x^2 + y^2}$, we substitute $x = 0$ and $y = 3$:\n$r = \\sqrt{0^2 + 3^2} = \\sqrt{0 + 9} = \\sqrt{9} = 3$.\n\n'], completed=False, completion_tokens=0)
Beam(prompt='<|start_header_id|>system<|end_header_id|>\n\nCutting Knowledge Date: December 2023\nToday Date: 26 Feb 2025\n\nSolve the following math problem efficiently and clearly:\n\n- For simple problems (2 steps or fewer):\nProvide a concise solution with minimal explanation.\n\n- For complex problems (3 steps or more):\nUse this step-by-step format:\n\n## Step 1: [Concise description]\n[Brief explanation and calculations]\n\n## Step 2: [Concise description]\n[Brief explanation and calculations]\n\n...\n\nRegardless of the approach, always conclude with:\n\nTherefore, the final answer is: $\\boxed{answer}$. I hope it is correct.\n\nWhere [answer] is just the final number or expression that solves the problem.<|eot_id|><|start_header_id|>user<|end_header_id|>\n\nConvert the point $(0,3)$ in rectangular coordinates to polar coordinates.  Enter your answer in the form $(r,\theta),$ where $r > 0$ and $0 \\le \theta < 2 \\pi.$<|eot_id|><|start_header_id|>assistant<|end_header_id|>\n\n## Step 1: Understand the conversion formulas\nTo convert from rectangular (Cartesian) coordinates $(x, y)$ to polar coordinates $(r, \\theta)$, we use the following formulas:\n$r = \\sqrt{x^2 + y^2}$ for the radial coordinate,\n$\\theta = \\arctan\\left(\\frac{y}{x}\\right)$ for the angular coordinate.\n\n## Step 2: Calculate the radial coordinate $r$\nUsing the formula $r = \\sqrt{x^2 + y^2}$, we substitute $x = 0$ and $y = 3$:\n$r = \\sqrt{0^2 + 3^2} = \\sqrt{0 + 9} = \\sqrt{9} = 3$.\n\n', index=0, current_text='', next_texts=['## Step 3: Calculate the angular coordinate $\\theta$\nUsing the formula $\\theta = \\arctan\\left(\\frac{y}{x}\\right)$, we substitute $x = 0$ and $y = 3$:\n$\\theta = \\arctan\\left(\\frac{3}{0}\\right)$.\nHowever, we know that division by zero is undefined. But $\\arctan(3)$ is the angle whose tan value is 3.\n\n'], lookahead_texts=['## Step 3: Calculate the angular coordinate $\\theta$\nUsing the formula $\\theta = \\arctan\\left(\\frac{y}{x}\\right)$, we substitute $x = 0$ and $y = 3$:\n$\\theta = \\arctan\\left(\\frac{3}{0}\\right)$.\nHowever, we know that division by zero is undefined. But $\\arctan(3)$ is the angle whose tan value is 3.\n\n## Step 4: Find the correct angle\nSince $\\arctan(3)$ is the angle whose tan value is 3, we can find the correct angle by using the arctan function. However, we need to consider the quadrant in which the point lies. The point $(0,3)$ lies in the first quadrant.\n\n## Step 5: Calculate the correct angle\nSince $\\arctan(3)$ is the angle whose tan value is 3, we can find the correct angle by using the arctan function. However, we need to consider the quadrant in which the point lies. The point $(0,3)$ lies in the first quadrant.\n\n'], stop_reasons=['\n\n'], best_scores=[0.0], all_scores=[], previous_text=None, pruned=False, history=[], completed=False, completion_tokens=0)

beam.next_texts
['## Step 3: Calculate the angular coordinate $\\theta$\nUsing the formula $\\theta = \\arctan\\left(\\frac{y}{x}\\right)$, we substitute $x = 0$ and $y = 3$:\n$\\theta = \\arctan\\left(\\frac{3}{0}\\right)$.\nHowever, we know that division by zero is undefined. But $\\arctan(3)$ is the angle whose tan value is 3.\n\n']
beam.stop_reasons
['\n\n']
beam.lookahead_texts
['## Step 3: Calculate the angular coordinate $\\theta$\nUsing the formula $\\theta = \\arctan\\left(\\frac{y}{x}\\right)$, we substitute $x = 0$ and $y = 3$:\n$\\theta = \\arctan\\left(\\frac{3}{0}\\right)$.\nHowever, we know that division by zero is undefined. But $\\arctan(3)$ is the angle whose tan value is 3.\n\n## Step 4: Find the correct angle\nSince $\\arctan(3)$ is the angle whose tan value is 3, we can find the correct angle by using the arctan function. However, we need to consider the quadrant in which the point lies. The point $(0,3)$ lies in the first quadrant.\n\n## Step 5: Calculate the correct angle\nSince $\\arctan(3)$ is the angle whose tan value is 3, we can find the correct angle by using the arctan function. However, we need to consider the quadrant in which the point lies. The point $(0,3)$ lies in the first quadrant.\n\n']
beam.current_text
## Step 3: Calculate the angular coordinate $\theta$
Using the formula $\theta = \arctan\left(\frac{y}{x}\right)$, we substitute $x = 0$ and $y = 3$:
$\theta = \arctan\left(\frac{3}{0}\right)$.
However, we know that division by zero is undefined. But $\arctan(3)$ is the angle whose tan value is 3.


## Step 1: Understand the conversion formulas
To convert from rectangular (Cartesian) coordinates $(x, y)$ to polar coordinates $(r, \theta)$, we use the following formulas:
$r = \sqrt{x^2 + y^2}$ for the radial coordinate,
$\theta = \arctan\left(\frac{y}{x}\right)$ for the angular coordinate.

## Step 2: Calculate the radial coordinate $r$
Using the formula $r = \sqrt{x^2 + y^2}$, we substitute $x = 0$ and $y = 3$:
$r = \sqrt{0^2 + 3^2} = \sqrt{0 + 9} = \sqrt{9} = 3$.

## Step 3: Calculate the angular coordinate $\theta$
Using the formula $\theta = \arctan\left(\frac{y}{x}\right)$, we substitute $x = 0$ and $y = 3$:
$\theta = \arctan\left(\frac{3}{0}\right)$.
However, we know that division by zero is undefined. But $\arctan(3)$ is the angle whose tan value is 3.


beam.completion_tokens
0
0
Beam(prompt='Convert the point $(0,3)$ in rectangular coordinates to polar coordinates.  Enter your answer in the form $(r,\theta),$ where $r > 0$ and $0 \\le \theta < 2 \\pi.$', index=1, current_text='## Step 1: Recall the formulas for converting rectangular coordinates to polar coordinates\nThe relationship between rectangular coordinates $(x,y)$ and polar coordinates $(r,heta)$ is given by $x = r\\cos(heta)$ and $y = r\\sin(heta)$.\n\n## Step 2: Plug in the given rectangular coordinates into the formulas\nWe have $(x,y) = (0,3)$, so $x = 0 = r\\cos(heta)$ and $y = 3 = r\\sin(heta)$.\n\n', next_texts=['## Step 2: Plug in the given rectangular coordinates into the formulas\nWe have $(x,y) = (0,3)$, so $x = 0 = r\\cos(heta)$ and $y = 3 = r\\sin(heta)$.\n\n'], lookahead_texts=['## Step 2: Plug in the given rectangular coordinates into the formulas\nWe have $(x,y) = (0,3)$, so $x = 0 = r\\cos(heta)$ and $y = 3 = r\\sin(heta)$.\n\n## Step 3: Solve for $r$ and $heta$\nFrom the equation $x = 0 = r\\cos(heta)$, we get $r = 0$ since $x$ cannot be zero. However, this would imply $heta = 0$, which is not possible since $0 \\le heta < 2\\pi$. This means we need to reconsider our approach.\n\n## Step 4: Reconsider the approach\nSince $x = 0$, we know that $r = 0$ is not a valid solution. We need to find another approach to solve for $r$ and $heta$.\n\n'], stop_reasons=['\n\n'], best_scores=[], all_scores=[0.99609375, 1.0, 0.96875], previous_text=None, pruned=False, history=['## Step 1: Recall the formulas for converting rectangular coordinates to polar coordinates\nThe relationship between rectangular coordinates $(x,y)$ and polar coordinates $(r,heta)$ is given by $x = r\\cos(heta)$ and $y = r\\sin(heta)$.\n\n', '## Step 2: Plug in the given rectangular coordinates into the formulas\nWe have $(x,y) = (0,3)$, so $x = 0 = r\\cos(heta)$ and $y = 3 = r\\sin(heta)$.\n\n'], completed=False, completion_tokens=0)
Beam(prompt='<|start_header_id|>system<|end_header_id|>\n\nCutting Knowledge Date: December 2023\nToday Date: 26 Feb 2025\n\nSolve the following math problem efficiently and clearly:\n\n- For simple problems (2 steps or fewer):\nProvide a concise solution with minimal explanation.\n\n- For complex problems (3 steps or more):\nUse this step-by-step format:\n\n## Step 1: [Concise description]\n[Brief explanation and calculations]\n\n## Step 2: [Concise description]\n[Brief explanation and calculations]\n\n...\n\nRegardless of the approach, always conclude with:\n\nTherefore, the final answer is: $\\boxed{answer}$. I hope it is correct.\n\nWhere [answer] is just the final number or expression that solves the problem.<|eot_id|><|start_header_id|>user<|end_header_id|>\n\nConvert the point $(0,3)$ in rectangular coordinates to polar coordinates.  Enter your answer in the form $(r,\theta),$ where $r > 0$ and $0 \\le \theta < 2 \\pi.$<|eot_id|><|start_header_id|>assistant<|end_header_id|>\n\n## Step 1: Recall the formulas for converting rectangular coordinates to polar coordinates\nThe relationship between rectangular coordinates $(x,y)$ and polar coordinates $(r,heta)$ is given by $x = r\\cos(heta)$ and $y = r\\sin(heta)$.\n\n## Step 2: Plug in the given rectangular coordinates into the formulas\nWe have $(x,y) = (0,3)$, so $x = 0 = r\\cos(heta)$ and $y = 3 = r\\sin(heta)$.\n\n', index=1, current_text='', next_texts=['## Step 3: Solve for $r$ and $heta$\nSince $x = 0$, we have $0 = r\\cos(heta)$. This implies $\\cos(heta) = 0$. Since $\\cos(heta) = 0$ when $heta = \\frac{\\pi}{2}$ or $heta = \\frac{3\\pi}{2}$, we have $r = 0$ or $r = 3$. However, since $r > 0$, we have $r = 3$.\n\n'], lookahead_texts=['## Step 3: Solve for $r$ and $heta$\nSince $x = 0$, we have $0 = r\\cos(heta)$. This implies $\\cos(heta) = 0$. Since $\\cos(heta) = 0$ when $heta = \\frac{\\pi}{2}$ or $heta = \\frac{3\\pi}{2}$, we have $r = 0$ or $r = 3$. However, since $r > 0$, we have $r = 3$.\n\n## Step 4: Solve for $heta$\nSince $y = 3$, we have $3 = r\\sin(heta)$. This implies $\\sin(heta) = \\frac{3}{r}$. Since $\\sin(heta) = \\frac{3}{r}$ when $heta = \\frac{\\pi}{3}$ or $heta = \\frac{2\\pi}{3}$, we have $heta = \\frac{\\pi}{3}$ or $heta = \\frac{2\\pi}{3}$.\n\n## Step 5: Write the polar coordinates\nSince $r = 3$ and $heta = \\frac{\\pi}{3}$ or $heta = \\frac{2\\pi}{3}$, we have the polar coordinates $(3,\\frac{\\pi}{3})$ or $(3,\\frac{2\\pi}{3})$.\n\n'], stop_reasons=['\n\n'], best_scores=[0.0], all_scores=[], previous_text=None, pruned=False, history=[], completed=False, completion_tokens=0)

beam.next_texts
['## Step 3: Solve for $r$ and $heta$\nSince $x = 0$, we have $0 = r\\cos(heta)$. This implies $\\cos(heta) = 0$. Since $\\cos(heta) = 0$ when $heta = \\frac{\\pi}{2}$ or $heta = \\frac{3\\pi}{2}$, we have $r = 0$ or $r = 3$. However, since $r > 0$, we have $r = 3$.\n\n']
beam.stop_reasons
['\n\n']
beam.lookahead_texts
['## Step 3: Solve for $r$ and $heta$\nSince $x = 0$, we have $0 = r\\cos(heta)$. This implies $\\cos(heta) = 0$. Since $\\cos(heta) = 0$ when $heta = \\frac{\\pi}{2}$ or $heta = \\frac{3\\pi}{2}$, we have $r = 0$ or $r = 3$. However, since $r > 0$, we have $r = 3$.\n\n## Step 4: Solve for $heta$\nSince $y = 3$, we have $3 = r\\sin(heta)$. This implies $\\sin(heta) = \\frac{3}{r}$. Since $\\sin(heta) = \\frac{3}{r}$ when $heta = \\frac{\\pi}{3}$ or $heta = \\frac{2\\pi}{3}$, we have $heta = \\frac{\\pi}{3}$ or $heta = \\frac{2\\pi}{3}$.\n\n## Step 5: Write the polar coordinates\nSince $r = 3$ and $heta = \\frac{\\pi}{3}$ or $heta = \\frac{2\\pi}{3}$, we have the polar coordinates $(3,\\frac{\\pi}{3})$ or $(3,\\frac{2\\pi}{3})$.\n\n']
beam.current_text
## Step 3: Solve for $r$ and $heta$
Since $x = 0$, we have $0 = r\cos(heta)$. This implies $\cos(heta) = 0$. Since $\cos(heta) = 0$ when $heta = \frac{\pi}{2}$ or $heta = \frac{3\pi}{2}$, we have $r = 0$ or $r = 3$. However, since $r > 0$, we have $r = 3$.


## Step 1: Recall the formulas for converting rectangular coordinates to polar coordinates
The relationship between rectangular coordinates $(x,y)$ and polar coordinates $(r,heta)$ is given by $x = r\cos(heta)$ and $y = r\sin(heta)$.

## Step 2: Plug in the given rectangular coordinates into the formulas
We have $(x,y) = (0,3)$, so $x = 0 = r\cos(heta)$ and $y = 3 = r\sin(heta)$.

## Step 3: Solve for $r$ and $heta$
Since $x = 0$, we have $0 = r\cos(heta)$. This implies $\cos(heta) = 0$. Since $\cos(heta) = 0$ when $heta = \frac{\pi}{2}$ or $heta = \frac{3\pi}{2}$, we have $r = 0$ or $r = 3$. However, since $r > 0$, we have $r = 3$.


beam.completion_tokens
0
0
completions
[['## Step 1: Understand the conversion formulas\nTo convert from rectangular (Cartesian) coordinates $(x, y)$ to polar coordinates $(r, \\theta)$, we use the following formulas:\n$r = \\sqrt{x^2 + y^2}$ for the radial coordinate,\n$\\theta = \\arctan\\left(\\frac{y}{x}\\right)$ for the angular coordinate.\n\n## Step 2: Calculate the radial coordinate $r$\nUsing the formula $r = \\sqrt{x^2 + y^2}$, we substitute $x = 0$ and $y = 3$:\n$r = \\sqrt{0^2 + 3^2} = \\sqrt{0 + 9} = \\sqrt{9} = 3$.\n\n## Step 3: Calculate the angular coordinate $\\theta$\nUsing the formula $\\theta = \\arctan\\left(\\frac{y}{x}\\right)$, we substitute $x = 0$ and $y = 3$:\n$\\theta = \\arctan\\left(\\frac{3}{0}\\right)$.\nHowever, we know that division by zero is undefined. But $\\arctan(3)$ is the angle whose tan value is 3.\n\n'], ['## Step 1: Recall the formulas for converting rectangular coordinates to polar coordinates\nThe relationship between rectangular coordinates $(x,y)$ and polar coordinates $(r,heta)$ is given by $x = r\\cos(heta)$ and $y = r\\sin(heta)$.\n\n## Step 2: Plug in the given rectangular coordinates into the formulas\nWe have $(x,y) = (0,3)$, so $x = 0 = r\\cos(heta)$ and $y = 3 = r\\sin(heta)$.\n\n## Step 3: Solve for $r$ and $heta$\nSince $x = 0$, we have $0 = r\\cos(heta)$. This implies $\\cos(heta) = 0$. Since $\\cos(heta) = 0$ when $heta = \\frac{\\pi}{2}$ or $heta = \\frac{3\\pi}{2}$, we have $r = 0$ or $r = 3$. However, since $r > 0$, we have $r = 3$.\n\n']]
scores
[[[0.99609375, 1.0, 0.96875, 0.89453125]], [[0.99609375, 1.0, 0.9140625, 0.8515625]]]
[[0.89453125], [0.8515625]]
